%--------------------------------------------------------------------%
%
% Title         : Postgraduate Thesis LaTex template
% Author        : Ravi Vendra Rishika <ravi.vendra.rishika@gmail.com>
%
% Page          : Chapter 1
%
% It is developed especially for postgraduate students of :
%   Department of Informatics
%   Faculty of Intelligent Electrical and Informatics Technology
%   Sepuluh Nopember Institute of Technology (ITS)
%   Surabaya, Indonesia.
%
% This LaTex template is intended to make students easier
% to write master's degree thesis in LaTex using specific
% Department of Informatics of ITS' format.
%
%--------------------------------------------------------------------%

\chapter{PENDAHULUAN}

Pada bab ini dijelaskan mengenai beberapa hal dasar dalam pembuatan proposal penelitian, yang meliputi sebagai berikut : Latar belakang, Perumusan masalah, Tujuan dan manfaat penelitian, Batasan penelitian, dan Kontribusi penelitian.

\section{Latar Belakang}

\blindtext

\section{Rumusan Masalah}

Berdasarkan latar belakang di atas, maka rumusan masalah yang akan dibahas di dalam penelitian ini adalah sebagai berikut :

\begin{enumerate}
    \item Pertanyaan ke-1 ?
    \item Pertanyaan ke-2 ?
    \item Pertanyaan ke-3 ?
\end{enumerate}

\section{Tujuan dan Manfaat Penelitian}

\blindtext

\section{Batasan Penelitian}

Untuk memfokuskan permasalahan di dalam penelitian ini, terdapat beberapa batasan masalah yang digunakan terkait topik penelitian ini :

\begin{enumerate}
    \item Batasan Penelitian ke-1.
    \item Batasan Penelitian ke-2.
    \item Batasan Penelitian ke-3.
\end{enumerate}

\section{Kontribusi Penelitian}

Kontribusi yang akan dilakukan oleh peneliti adalah sebagai berikut :

\begin{enumerate}
    \item Kontribusi Penelitian ke-1.
    \item Kontribusi Penelitian ke-2.
    \item Kontribusi Penelitian ke-3.
\end{enumerate}
