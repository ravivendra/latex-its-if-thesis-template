%--------------------------------------------------------------------%
%
% Title         : Postgraduate Thesis LaTex template
% Author        : Ravi Vendra Rishika <ravi.vendra.rishika@gmail.com>
%
% Page          : Chapter 1
%
% It is developed especially for postgraduate students of :
%   Department of Informatics
%   Faculty of Intelligent Electrical and Informatics Technology
%   Institut Teknologi Sepuluh Nopember (ITS)
%   Surabaya, Indonesia.
%
% This LaTex template is intended to make students easier
% to write master's degree thesis in LaTex using specific
% Department of Informatics of ITS' format.
%
%--------------------------------------------------------------------%

\chapter{PENDAHULUAN}
\label{sec:pendahuluan}

Pada bab ini dijelaskan mengenai beberapa hal dasar dalam penelitian dan penulisan laporan tesis, meliputi: (1) latar belakang, (2) perumusan masalah, (3) tujuan penelitian, (4) batasan penelitian, (5) manfaat penelitian, dan (6) kontribusi penelitian.

\section{Latar Belakang}

\blindtext \citep{bhati2020review}

\section{Rumusan Masalah}

Berdasarkan latar belakang di atas, maka rumusan masalah yang akan dibahas di dalam penelitian ini adalah sebagai berikut:

\begin{enumerate}
    \item Pertanyaan ke-1 ?
    \item Pertanyaan ke-2 ?
    \item Pertanyaan ke-3 ?
\end{enumerate}

\section{Tujuan Penelitian}

\blindtext

\section{Batasan Penelitian}

Untuk memfokuskan permasalahan di dalam penelitian ini, terdapat beberapa batasan masalah yang digunakan terkait topik penelitian ini :

\begin{enumerate}

    \item Batasan Penelitian ke-1.
    \item Batasan Penelitian ke-2.
    
\end{enumerate}

\section{Manfaat Penelitian}

\blindtext

\section{Kontribusi Penelitian}

Kontribusi yang akan dilakukan oleh peneliti adalah sebagai berikut :

\begin{enumerate}

    \item Kontribusi Penelitian ke-1.
    \item Kontribusi Penelitian ke-2.
    
\end{enumerate}
