%--------------------------------------------------------------------%
%
% Title         : Postgraduate Thesis LaTex template
% Author        : Ravi Vendra Rishika <ravi.vendra.rishika@gmail.com>
%
% Page          : Approval
%
% It is developed especially for postgraduate students of :
%   Department of Informatics
%   Faculty of Intelligent Electrical and Informatics Technology
%   Institut Teknologi Sepuluh Nopember (ITS)
%   Surabaya, Indonesia.
%
% This LaTex template is intended to make students easier
% to write master's degree thesis in LaTex using specific
% Department of Informatics of ITS' format.
%
%--------------------------------------------------------------------%

\clearpage

\addcontentsline{toc}{chapter}{LEMBAR PENGESAHAN TESIS}

\begin{center}

    \smallskip

    \large \bfseries \MakeUppercase{
        {Lembar Pengesahan Tesis}
    }

    \vspace{25pt}
    
    \normalsize \bfseries
    
    \begin{spacing}{1.1}

        \begin{center}
            \setlength{\tabcolsep}{5pt}
            \begin{tabular}{c{1.00\linewidth}}
                Tesis disusun untuk memenuhi salah satu syarat memperoleh gelar \\
                \postgraduateTitle \\
                di \\
                \postgraduateUniversity \\
            \end{tabular}
        \end{center}

        \vspace{10pt}
    
        \begin{center}
            \setlength{\tabcolsep}{5pt}
            \begin{tabular}{c{1.00\linewidth}}
                Oleh: \\
                \authorName \\
                NRP: {\authorNRP} \\
            \end{tabular}
        \end{center}

        \vspace{10pt}
    
        \begin{center}
            \setlength{\tabcolsep}{5pt}
            \begin{tabular}{p{0.25\linewidth} p{0.25\linewidth} p{0.50\linewidth}}
                & Tanggal Ujian       & : {\finalExamDate} \\
                & Periode Wisuda      & : {\finalGraduationPeriod} \\
            \end{tabular}
        \end{center}

        \vspace{10pt}
    
        \begin{center}
            \setlength{\tabcolsep}{5pt}
            \begin{tabular}{p{0.03\linewidth} p{0.70\linewidth} p{0.30\linewidth}}
                \multicolumn{3}{c}{Disetujui oleh:} \\
                & & \\
                \multicolumn{3}{c}{Pembimbing:} \\
                1.  & {\firstSupervisorName}        & \\
                    & NIP: {\firstSupervisorNIP}    & \rule{3cm}{0.01mm} \\[15pt]
                2.  & {\secondSupervisorName}       & \\
                    & NIP: {\secondSupervisorNIP}   & \rule{3cm}{0.01mm} \\
                & & \\
                \multicolumn{3}{c}{Penguji:} \\
                1.  & {\firstExaminerName}          & \\
                    & NIP: {\firstExaminerNIP}      & \rule{3cm}{0.01mm} \\[15pt]
                2.  & {\secondExaminerName}         & \\
                    & NIP: {\secondExaminerNIP}     & \rule{3cm}{0.01mm} \\[15pt]
                3.  & {\thirdExaminerName}          & \\
                    & NIP: {\thirdExaminerNIP}      & \rule{3cm}{0.01mm} \\
            \end{tabular}
        \end{center}

        \vspace{10pt}

        \begin{center}
            \setlength{\tabcolsep}{5pt}
            \begin{tabular}{c{1.00\linewidth}}
                Kepala {\postgraduateDepartment} \\
                {\postgraduateFaculty} \\
                \\
                \\
                \\
                \underline{\chiefInformaticsDepartmentName} \\
                NIP: {\chiefInformaticsDepartmentNIP} \\
            \end{tabular}
        \end{center}
    
        % \begin{center}
        %     \setlength{\tabcolsep}{5pt}
        %     \begin{tabular}{c{1.00\linewidth}}
        %         Direktur Pascasarjana dan Pengembangan Akademik \\
        %         \\
        %         \\
        %         \\
        %         \underline{\academicPostGraduateDirectorName} \\
        %         NIP: {\academicPostGraduateDirectorNIP} \\
        %     \end{tabular}
        % \end{center}

    \end{spacing}

\end{center}

\clearpage
