%--------------------------------------------------------------------%
%
% Title         : Postgraduate Thesis LaTex template
% Author        : Ravi Vendra Rishika <ravi.vendra.rishika@gmail.com>
%
% Page          : Forewords
%
% It is developed especially for postgraduate students of :
%   Department of Informatics
%   Faculty of Intelligent Electrical and Informatics Technology
%   Institut Teknologi Sepuluh Nopember (ITS)
%   Surabaya, Indonesia.
%
% This LaTex template is intended to make students easier
% to write master's degree thesis in LaTex using specific
% Department of Informatics of ITS' format.
%
%--------------------------------------------------------------------%

\clearpage

\addcontentsline{toc}{chapter}{KATA PENGANTAR}

% \chapter*{Kata Pengantar}
\section*{\large \centering \MakeUppercase 
    {Kata Pengantar}
}

\begin{center}

    \justifying \normalsize

    Puji dan syukur ke hadirat Allah SWT atas segala limpahan nikmat dan rahmat-Nya sehingga penulis dapat menyelesaikan penelitian tesis dengan judul “{\titleID}”.

    Tujuan dari penulisan tesis ini adalah untuk melengkapi salah satu syarat dalam mencapai derajat {\postgraduateTitle} di {\postgraduateDepartment}, {\postgraduateFaculty} (FT-EIC), {\postgraduateUniversity} (ITS), {\postgraduateCity}, Indonesia. 

    Dalam penyusunan tesis ini tidak terlepas dari pihak-pihak yang memberikan dukungan baik secara materiil maupun non-materiil, memberikan masukan, arahan, kritik dan saran sehingga penulisan ini dapat diselesaikan dengan optimal. Pada kesempatan ini, penulis ingin mengucapkan terima kasih kepada:
    % yang berjudul “{\titleID}”

    \begin{enumerate}

        \item .....
        \item .....
        \item .....

    \end{enumerate}

    Semoga Allah SWT selalu melindungi dan memberikan rahmat-Nya kepada pihak-pihak yang disebutkan di atas.

    Penulis sangat menyadari bahwa penulisan ini tidaklah luput dari kesalahan dan kekurangan, oleh karena itu penulis mengharapkan kritik dan saran yang bersifat membangun dari berbagai pihak agar dapat memberikan semangat baru bagi penulis untuk lebih baik pada kesempatan penulisan maupun penelitian berikutnya.

    Akhir kata, penulis mengharapkan semoga hasil dari penulisan dan penelitian ini dapat memberikan informasi yang bermanfaat bagi para pembaca dan dapat berkontribusi pada pendidikan di Indonesia.

    \begin{table}[h]
        \centering

        \begin{tabular}{p{8cm} c}
            & {\postgraduateCity}, {\writingDate} \\
            & \\
            & \\
            & \\
            & {\authorName} \\
            & NRP: {\authorNRP}
        \end{tabular}
    \end{table}

\end{center}

\clearpage
